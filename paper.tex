\documentclass[cmfonts]{witpress}
\usepackage{todonotes}
\bibliographystyle{witpress}


\begin{document}


\title{Improvement of crash forces in structures using optimization tools.}

\author{L. E. Romera, M. Costas, J. Paz, J. D\'iaz, S. Hern\'andez.}

\address{Structural Mechanics Group, Universidade da Coru\~na, Spain.}

\maketitle

\begin{abstract}
This work describes an investigation on the structural optimization of the crash response of two structural models: a car model and an airplane fuselage model, where the force pulse due to frontal impact or hard landing wanted to be kept as stable as possible while reducing the mass of the energy absorbing systems.  The objective functions were the variance of the force-time curves produced in the impact tests and the mass of the vehicles, both of them being minimized in a multi-objective approach using metamodeling on FE models and genetic evolutionary algorithms. Finite element models were subjected to a frontal impact test against a simplified rigid wall in the case of the car model, or to hardlanding impact for the fuselage model. Force-time curves was obtained from the analysis and suitably filtered. The objective functions were calculated as the variance of this force-time curves and the total structural mass. The objective was to obtain a force-time curve which was as close as possible to the ideal dissipator curve, with lower values of the peaks of acceleration experienced by the occupants; and to minimize the mass for fuel saving and environmental reasons. To that end, optimization strategies were planned carefully to deal with problems which are typical in crashworthiness optimization like expensive computation times and numerical noise.
\end{abstract}\\
\emph{Keywords: crashworthiness, injuries reduction, size optimization, surrogate models.}

\section{Blabla}
\cite{funke84}
\todo{Quitar usepackage todonotes}
\todo{Poner bien las referencias (citet pero que funcione)}
\section{Blabla}
\cite{brebbia84}


\bibliography{./references/references}

\end{document}
