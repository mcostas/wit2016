\documentclass[cmfonts]{witpress}
\begin{document}
\bibliographystyle{witpress}

\title{Axial crushing of aluminum extrusions filled with PET foam and GFRP. Numerical modeling and multiobjective optimization.}

\author{M. Costas, L. Romera}
% \author{D. Morin}
% \author{M. Langseth}
% \author{J. D\'iaz}
% \author{L. Romera}

\address{Structural Mechanics Group, School of Civil Engineering. Universidade da Coru\~na. Campus de Elvi\~na, 15071, A Coru\~na, Spain}

\maketitle

\begin{abstract}
This investigation constitutes the continuation of a previous experimental study by the authors on the axial crushing of aluminum extrusions filled with PET foam and a GFRP skeleton. Herein, a finite element model was calibrated with the results obtained in a material testing campaign using appropriate constitutive equations. Customary models were used for the material behavior of the aluminum alloy (a J2--plasticity model) and the PET foam (Deshpande and Fleck's model). Regarding the short-fibers GFRP, a Voce plasticity model was preferred to usual viscoelastic models for simplicity, as long as the results were satisfying. After a successful validation of the finite element model, the filled aluminum extrusion was subjected to a structural optimization to achieve the best crash performance. Three relevant design variables were selected: the thickness of the outer aluminum cylinder, the thickness of the GFRP and the density of the PET foam, the latter being related to its crushing strength. Given the high computational cost of each finite element model, a multi-adaptive regression splines metamodel was fitted to a large-scale sampling. Optimum pairs were obtained for the absorbed energy, the specific energy absorption, the peak load and the component's mass; stating the relative contribution of each design variable to the crashworthiness of the crash box and enabling the choice of a balanced optimum design.
\end{abstract}


\end{document}
