\chapter{State of the Art}
\label{Chapter1}
% Las referencias ahora están en un nuevo archivo: \texttt{pt_luis_pire.bib}.
%----------------------------------------------------------

% Acronimos, sacado de Wikibooks
%\usepackage{glossaries}
\newacronymentry{FEM}{FEM}{finite element method}
\newacronymentry{XFEM}{XFEM}{extended finite element method}
\newacronymentry{SLJ}{SLJ}{single-lap joint}

\newacronymentry{SEA}{SEA}{specific energy absorption}
\newacronymentry{Ea}{$E_a$}{energy absorption} % Revise this!!!!!

\newacronymentry{SW}{SW}{spot-weld}
\newacronymentry{LW}{LW}{laser-weld}

The study of adhesive's behaviour subjected to impact loads through the \gls{FEM} is becoming more interesting since the popularization of this type of bonding in the last few years in many industries, such as the automotive or the aerospatial ones \citep{Wu2006, Greve2007, Grant2009, Scattina2011, Kadioglu2014, SernaMoreno2015}. Adhesive's properties let manufacture highly-resistant light-weight joints, with good results on fatigue. Hybrid bonding has also become interesting as a way to improve weld unions through the use of adhesives.

The \gls{FEM} has been widelly used to model adhesive bonding behaviour \citep{Sato2000, Alfano2001, Kihara2003, Vaidya2006, Wu2006, Hou2008, Grant2009, Peroni2009, Sadowski2010, Scattina2011, Sadowski2011, Liao2011, Yang2012}% check if others also did, just in case
, usually as a previous phase to laboratory experiences.

Many different tests and analysis have been carried out on the adhesives. \citet{Loureiro2010} tested stiff and flexible adhesives in order to compare their mechanical behaviour under impact, static loads, fatigue and damping. Damage has been tested \citep{Goglio2008, Kadioglu2014}, but more work in this field is still needed, as \citet{Greve2007} stated.

Design variations have also been tried, including the use of fillets \citep{Grant2009, Yang2012}, different thicknesses \citep{Grant2009}, or additives for the adhesive \citep{Vaidya2006}, several geometries for the joint, such as \glspl{SLJ} and T-joints \citep{Loureiro2010}, and for the adherends \citep{Sato2000}.

As \citet{Hou2008} pointed, the design of a crashworthiness element pursues to absorb the greatest amount of energy through elastic and plastic deformation \citep{Wu2006}. This also increases the peak force, which is considered one of the main reasons of biomechanical injury, becoming another key parameter in the design. The element's weight impacts on long-term costs, both economically and environmentally. \Gls{SEA} has been used as parameter by some authors \citep{Lee2006, Peroni2009, Scattina2011} for measuring the element's effectiveness on absorving energy, rather than the total \gls{Ea}. A compromise between these parameters has to be achieved.

The design and analisis of crash boxes for the automotive industry is one of the main applications of this knowledge. \citet{Lee2006} showed the adhesive's better performance if compared to \gls{SW} in terms of \gls{SEA}. These two bonding methods were also tested, toghether with \gls{LW}, by \citet{Peroni2009}, who also included different box section geometries in their studies. \citet{Sadowski2010, Sadowski2011} modelled hybrid unions with the use of rivets at different scales, and eventually tested crash boxes with hybrid bonds.

%----------------------------------------------------------

\citet{Kihara2003} made a two-dimensional analysis through the FEM in order to model their proposed experiment before taking it to the laboratory. On their experiments, they took special attention on the crack of the adhesive layer.

\citet{Vaidya2006} continued this studies by deeply analysing, numerically and experimentally, lap joints casted with adhesives on impact. They analysed the adhesive's response under different load types, including a peel and transverse loads combination, and tried the addition of nanoclay to the adhesive, resulting in an approximate 20\% Young's modulus and a decrease of the ultimate failure strain of one third of the original.

\citet{Liao2011} modeled a single-lap adhesive joint under impact tensile loads. They used a 3D finite element model to analyse the stresses, and to evaluate the total strength the joint could handle. They deepened the most in the distribution of stresses in the adhesive, finding the most stressed regions and analysing the time needed by the stresses to reach their peak related to the adhesive stiffness.

\citet{Loureiro2010} tested a stiff adhevive and a flexible one in order to compare their mechanical behaviour. They analysed single-lap joints and T-joints under different types of loads, including impact, but also static loads, fatigue and damping. They found that stiff adhesives had a bigger impact failure load than their counterpart, although they are comparable. In single-lap joints, stiff adhesives produced adherend yielding, which resulted in failure and the cause of the failure load increase. Flexible adhesives showed a bigger distribution in time of load and adherend yielding in T-joints.

\citet{Kadioglu2014} made a study on adhesive tape for its use in automobiles. They showed their experimental results under impact loads for this type of bonding, comparing them to quasi-static results. This tests were made with a Charpy-like pendulum, showing apptitude for testing. They concluded that results vary with little speed on impact variation, at room temperature.

\citet{Goglio2008} also used a Charpy-like pendulum to test the rupture of joints casted with structural adhesives under impact loads producing different strees combinations. To achieve this, they analysed several joints with different lap length, thicknesses, etc. They found a big increase of the maximum stress values when testing on impact, compared to the static results, specially in the shear component.

\citet{Grant2009} also studied the use of adhesives in the automotive industry. They tried different adhesive thicknesses and studied their behaviour: in the case of bending, tests showed the joint strength was independent of this in spite of becoming stiffer; when joints are under tension, the bigger the thickness of the adhesive, the smaller the joint strength. They tried a 45\degree fillet, which showed stronger joints, specially when the adhesive becomes thicker. They also proposed a failure criterion which takes into account the tensile load and the bending moment suffered.

\citet{Sato2000} tried several types of joints through the FEM: single-lap, tapered lap, and scarf joint. Different results were accomplished, showing that tapered lap joints distribute certain types of loads in a more uniform way, compared to single-lap joints. They also checked the differences between static and dynamic tests. A five-element Voigt model was used to approximate the viscoelastic behaviour of the adhesive, which allowed to analyse the stress distribution in time and the stress wave propagation.

The energy absorption of this last mechanism in elastic-plastic materials was studied by \citet{Wu2006}, and compared it to the other mechanism of dissipation: plastic deformation. This was carried out with a 2D-model with axisymmetry, as the impact conditions allowed to do so, and taking special attention on the contact modelling. The restitution coefficient was also taken into account, as a measure of the amount of dissipated energy, relating it to the impact velocity.

The design of a crashworthiness element pursues to absorb the greatest amount of energy through deformation but, as \citet{Hou2008} showed, this also increases the peak force, which is considered one of the main reasons of biomechanical injury, becoming another key parameter in the design. These authors also analysed several multi-cell sections (up to a quadruple-cell section), showing their differences in their response. They made a multi-objective optimization, and a single-objective optimization taking other objectives as constrains, comparing the results.

\citet{Lee2006} tested double hat-shaped specimens with spot-welded with similar materials, and adhesively bonded with steel and aluminium. They found that both specimens absorbed aproximatedly the same energy, although the aluminium-steel adhesively bonded one had a 37\% increase of its specific energy absorption (SEA), in spite of a decrease of the mean crush load. They also made a point on the collapse modes of the adhesive and of the other bonding methods tested.

Using carbon fiber reinforced plastic (CFRP) adherends, \citet{Wu2013} analysed the crashworthiness of this adhesive joints under transverse loads. They tested the overlap length and width on the joint load, stiffness, propagation displacement and absorbed energy on quasi-static experiments. On impact, they analysed the relation of force with deflection and with time, and of the energy absorption with time and with impact energy. They found that the bigger the peak absorbed energy, the more proportion of energy was finally absorbed. A part of this energy was rebounded to the system. They also found force drops in their experiments, due to the initiation of the adhesive crack.

\citet{Peroni2009} compared the traditional joining technique in the automotive industry (spot-welding) to adhesive and laser joints, in order to show the aptitude of this two novel bonding methods. They presented the main adhesive's advantages and drawbacks, remarking that structural adhesives have improved their peel stress resistance, which used to be one of their main problems. In spite of this, triggers and rivets were practiced into the test boxes in order to ensure an stable collapse. They presented five experimental configurations (two flanged and three unflanged), and the key problems of each, with focus on manufacturing difficulties. The results were satisfactory: adhesive bonding gives a better performance than spot-wielding, and is comparable with laser bonding, which is slightly stiffer.

\citet{Greve2007} proposed a highly specific model for thin-layer high-strength adhesives. Certain phenomena, such as the non-conservation of volumen through deformation, make these new models necessary. A fracture criterion was token into account for this modelling, although more work in this field is needed, as stated. Two different FEM were simulated: one with solid elements for the adhesive, and another with node-to-element modelling. Both gave good results when compared to the real model, although the former gave the best adjustment, while the later had a much lower computational cost.

%Ejemplo de cita parentética: However, some authors disagree with the previous statement \citep{Kadioglu2014}. Thus ...
%Falta Sadowski. Incluir tambien otros mencionados mas adelante?
