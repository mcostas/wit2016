\chapter{Introduction}
\label{Chapter0}

%----------------------------------------------------------

Technological advances of the last decades, together with the newest tendencies on vehicle designing, have led society to drive bigger and faster cars. As the mass and the speed of an object increase, also does the kinetic energy. Therefore, in case of having an accident, passengers would suffer a more energetic impact and, thus, a more damaging impact. Security becomes a main concern in the automobilistic industry as the potential damage of the vehicles increases.

Crashworthy elements are installed as a part of the vehicle's structure, being their function to absorb the most energy in case of impact.

This elements must have a large impact resistance, but other variables are also considered in their design process, such as its weight. A lighter piece implies lower costs in the long term economically and environmentally, as it causes a lower fuel consumption. Adhesive bonding, which has been being used for a long time in aeronautics because of its resistance and lightness, becomes an interesting candidate for crashworthy elements casting.