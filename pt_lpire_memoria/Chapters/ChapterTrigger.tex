\chapter{About triggering}
\label{ChapterTrigger}

% This chapter is a compendium of notes and thoughts about consequences of many of the changes that had beed proben during the development of this document. This includes many possible fabrication imperfections taken into account or design possibilities that turned out to be wrong.

Trigger design is key on a crash box conception. Its main purpose is to ease a certain collapse mechanism instead of another different one, usually through some defformation \citep{Abedrabbo2009, Scattina2011, Costas2013} or minor changes in geometry such as practicing some holes \citep{Peroni2009, Yamashita2013}.

Triggering an element often implies weakening a section or zone of it, as some part of the plastic defformation reserve is consumed or less resistant material is placed there. Although this seems a priori bad, in the case of crash boxes, a properly made trigger helps to achieve greater values of \gls{SEA}, resulting in lower chances of injure for passengers in case of impact.

In spite of that, minor differences in trigger or crash box design or manufacture may result in huge deviations from expected results. As an order of magnitude, if triggers don't work properly, \gls{SEA} may reduce its value to a half or even a third.

\section{Factors}%------------------------------------------------------------------------------------------------------------------------
The type of trigger is very important for the posterior collapse of the crash box, as certain triggers through impossed defformation may ease peeling in the union.

Global response of the crash box strongly depends on the adherends thickness. In this case, three plate thicknesses were tried out: $\SI{1.0}{\mm}$, $\SI{1.5}{\mm}$ and $\SI{2.0}{\mm}$. The $\SI{1.0}{\mm}$ thick plate produces defformation waves in all the crash box length simultaneously. In contrast, in the $\SI{1.5}{\mm}$ case, defformation waves happen sequentially. The $\SI{2.0}{\mm}$ thick profiles were found out to be too rigid, resulting a general bending of both plates in opposite directions and peeling the union.

\section{To avoid}%----------------------------------------------------------------------------------------------------------------------
The formation of two consecutive waves in the same direction (both in, or both out) may cause the crash box to become very rigid in that point due to some sort of membrane effect, and bending just next to that point.